\documentclass[11pt]{article}
\usepackage[top=1in, bottom=1.25in, left=1.25in, right=1.25in]{geometry}
\newcommand{\tabitem}{~~\llap{\textbullet}~~}
\usepackage[parfill]{parskip}
\usepackage{enumerate,amsmath,amsthm,amssymb,bbold}
\usepackage{minibox,graphicx,caption,booktabs,pdflscape,multirow,verbatim,subcaption,pdfpages,longtable}
\usepackage[shortlabels]{enumitem}
\usepackage[T1]{fontenc}
\usepackage[section]{placeins} %to keep graphs and tables in the very section to which they belong 
%-----------------------------------------------------------------------------
\begin{document}
\begin{center}
\framebox[\linewidth]{ 
	\minibox[c]{
	\Large Homework \#3 \\ \\
	Professor: Pat Kline \\ \\
	Students: Christina Brown, Sam Leone, Peter McCrory, Preston Mui
	}
}
\end{center}

\bigskip \textbf{Reweighting}

\bigskip \textbf{Matching}

\bigskip \textbf{2SLS}

\begin{enumerate}[(a)]

	\item The moment conditions defining the 2SLS estimator are
	\begin{align*}
		E \left[ Z_i \epsilon_i \right] &= 0 \\
		E \left[ Z_i^2 \epsilon_i \right] &= 0
	\end{align*}
	they imply each other, since functions of uncorrelated variables are uncorrelated with each other.

	\item The moment conditions defining the control function estimator are
	\begin{align*}
		E \left[ Z_i \epsilon_i \right] &= 0 \\
		E \left[ Z_i v_i \right] &= 0 \\
		E \left[ \epsilon_i | v_i \right] &= \rho \frac{\sigma_{\epsilon}}{\sigma_v} v_i \\
		E \left[ (Y_i - \beta_0 - \beta_1 X_i - \beta_2 X_i^2 - \rho \frac{\sigma_{\epsilon}}{\sigma_v} v_i)X_i \right] &= 0 \\
		E \left[ (Y_i - \beta_0 - \beta_1 X_i - \beta_2 X_i^2 - \rho \frac{\sigma_{\epsilon}}{\sigma_v} v_i)X_i^2 \right] &= 0 \\
		E \left[ (Y_i - \beta_0 - \beta_1 X_i - \beta_2 X_i^2 - \rho \frac{\sigma_{\epsilon}}{\sigma_v} v_i)v_i \right] &= 0
	\end{align*}

	\item The control function estimator relies on stronger conditions than the 2SLS estimator. The only condition in the 2SLS estimator is that the instrument, $Z$, is uncorrelated with the error term $\epsilon$. However, the control function estimator relies on specifying the exogenous relationship between $X$ and $Z$, in particular that the conditional expection of $X$ given $Z$ is linear.

	\item The control function would work better when $Z$ is weak, because it brings to bear more moment restrictions than the 2SLS estimator, whereas the 2SLS estimator only relies on the exogeneity of $Z$. %Should add more but not sure what

	\item

\end{enumerate}

\bigskip \textbf{Bootstrap OLS}

\bigskip \textbf{Bootstrap Probit}

\end{document}